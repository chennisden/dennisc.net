\documentclass{bounce}

\activepalette{bounce}{green}

\title{On Contest Creation}
\subtitle{A reflection of my math contest creation experiences}
\author{Dennis Chen}
\date{\today}

\begin{document}
\maketitle

\begin{abstract}
    This is a reflection of the contests I've created in the last two years. At the same time, it is meant to serve as a loose guide for anyone looking to create their own contest or refine their craft.
\end{abstract}

\section{How to start}

\subsection{What's your purpose?}

The mission of your contest isn't something you actively need to think about, but I think it's important to keep your contest purpose somewhere in your head because it informs your \emph{test compilation values}.

Take the MAT, for instance. Most contests have primary goals of being a) educational and b) fun, and sometimes this comes at the expense of problem quality. MAT instead focuses on \hyperref{https://myanimelist.net/manga/114745/Blue_Lock}{egoism}; we strive first and foremost to make a beautiful test with only standout problems. I like to believe that our contest is also educational and fun, but it is undeniable our focus on \emph{problem quality} supercedes most groups.\footnote{Actually, now that I think about it, MAT really is like Blue Lock. A contest full of standout problems and a team full of standout strikers. Perhaps I should use this analogy more often, at least in MAC.}

\section{Careful!}


\end{document}