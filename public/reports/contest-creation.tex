\documentclass[points=false]{bounce}

\activepalette{bounce}{green}

\title{On Contest Creation}
\subtitle{Thoughts on the First MAT}
\author{Dennis Chen}
\date{\today}

\begin{document}
\maketitle

\begin{abstract}
    This is a reflection of the contests I've created under Math Advance for the last two years, primarily the Summer 2021 MAT. At the same time, it is meant to serve as a loose guide for anyone looking to create their own contest or refine their craft.

    This report assumes you have some idea of what makes a problem worth doing and what makes a contest worth taking. In short, I assume you have \emph{taste}. It also assumes basic common sense, because my goal is not to make a document for the $95\%$ who can't figure out that they should announce their contest after they write it. It is for the contest organizers with real potential.
\end{abstract}

\section{How to start}

\subsection{What's your purpose?}

The mission of your contest isn't something you actively need to think about, but I think it's important to keep your contest purpose somewhere in your head because it informs your \emph{test compilation values}.

Take the MAT, for instance. Most contests have primary goals of being a) educational and b) fun, and sometimes this comes at the expense of problem quality. MAT instead focuses on \href{https://myanimelist.net/manga/114745/Blue_Lock}{egoism}; we strive first and foremost to make a beautiful test with only standout problems. I like to believe that our contest is also educational and fun, but it is undeniable our focus on \emph{problem quality} supercedes most groups.\footnote{Now that I think about it, MAT really is like Blue Lock. A contest full of standout problems and a team full of standout strikers. Perhaps I should use this analogy more often, at least in MAC.}

Even though none of these goals are at odds with each other, the difficulty of putting together a test compounds exponentially with each goal you have. So if you're not meeting your goals, you either have to tighten your focus, spend more time, or put less problems on the test.

Two of your contest series do not have to have the same philosophy. The MAT is produced like a work of art, meaning it is smaller and takes a lot longer to finish. The JMCs are meant for practice, so less emphasis is placed on egoism (making sure each individual problem is good) and moreso on cohesion (the test is good practice as a whole).

Your philosophy for a line of contests can also change between contests. For instance, the first MAT had a large emphasis on egoism. We wanted to put together the best test we could, and we made sure that the difficulty scale made sense on a \emph{relative} scale. But next time we're going to put together a contest whose difficulty makes more sense on an \emph{absolute} scale. This is especially because we're going to be moving locally, where the average contestant will not be as skilled.

\subsection{Getting people}

This part is the hardest, because you can't go out and send advertisements to everyone.
You need to be deliberative about who you pick, but at the same time,
your group has to form spontaneously.
This is because, at any point in time, people have different ideas on what they want to do,
and your job as a leader is to find people who want to write contests,
take that feeling, capture it, and magnify it.

In short, you have to find people without
trying too hard to look for them.
In fact, the rule of thumb I use
is that if I have to ask more than once,
they're not interested enough to be on the team anyways.
\footnote{I'm in a position to expect that enough people
will want to work with me to ask themselves.
So that's the rule I tend to go by,
especially since we already have plenty of people
working on Math Advance projects.}

If you can't find enough people,
then don't run your contest yet.
This doesn't mean you can't make preparations
or write problems on your own.
It just means that you should be patient.

I've seen a lot of people run crappy contests
with a couple of standout problems.
This is a real shame, because if only they'd waited longer,
they might've been able to put together something good,
rather than something mediocre.

The way you find people to work with is pretty simple.
Get to know more people.
That's how you get a wider base of people to draw from
that trust your judgment.

When you're working with people,
it's important to have a \emph{center of gravity}.
The chain of leadership needs to be clear, and it also needs to be intuitive. Arbitrarily selected leaders generally don't tend to work well.

It should be clear who the primary leader and representative of your group are. They don't have to be the same person -- for instance, I'm not the one sending most of the official emails in Math Advance -- but roles should naturally fit the people they are assigned to.

There's a reason I say roles fit people, instead of people fit roles.

It's a typical misconception that you should find the best person for a certain role, and repeat for however many roles you ave.
It gets further magnified by the official structures of clubs and non-profits, where they treat roles as permanent and people as interchangable parts.

It's roles, rather, that should be frequently rotated out as everyone's interests and circumstances change.

So don't try to look for people because you think they'd make a good secretary, or a good treasurer, or whatever. Look for people who seem interesting and are interested in contributing to your group. Even if only half of them do any serious work, it'll far outshine any robotic organization put together by placing people in roles, rather than giving roles to people.

\section{Test-setting}

\section{Logistics}

Here is the long and short of it. You need
\begin{enumerate}
    \item A website,
    \item sponsors,
    \item and outreach,
\end{enumerate}
\emph{in this order}. Most sponsors will ask for a website when evaluating your email, so you absolutely have to finish that first, though you can work on email drafts before then. You could start outreach before you get sponsors, but I strongly advise against this, particularly because a signup period of two or three weeks is ideal for online contests, and that is not enough time for sponsorship queries to be finalized.\footnote{I would imagine that the ideal signup period for in-person contests is about a month more, i.e. three or four weeks.}

What follows are specifics on sponsors and outreach, as well as some general tips on time management. If you want to learn how to make a website, read the docs, they'll teach you much better than I can.\footnote{Feel free to send me questions though, if you've done your own work.}

\subsection{Sponsors}

\subsection{Outreach}



\subsection{Time management}

By the time you publicly announce your contest, you should have everything finalized. Now, this is not always possible and sometimes you want to strategically push out a public announcement just to force some final checks to happen. But be aware this is a dangerous thing to do.

At no point should you announce a contest if there is no way you can finish on time.
Nor should decide early on that you can finish on time, especially not in the face of contradictory evidence. early.
Your reputation as a contest host is important,
and you should not throw it away by haphazardly announcing contests that you know you cannot hold.

Just as a sufficiently complex computer system should not be single-threaded, nor should your contest preparation be. I understand that it is hard to be confident your contest will even exist when you don't have a test composed or many problems written. But it is best to start handling logistics as soon as possible. That does not mean you should immediately begin contacting sponsors before you've even assembled a draft, but nor should you wait until you've checked the contest $50$ times to get your website started.\footnote{And if you \emph{do} check your contest $50$ times, you tend to grow complacent afterwards. So it doesn't do you a lot of good anyway.}

When it comes to scheduling, you want to host your contest in a period when interest is at its peak, especially if it's the inaugural contest. A couple of good times are early June, late July (before school is going to start but close enough to the start of school), and January/early February. Early and mid October may become good times as well, because the AMCs are moving to November. 

But besides that, don't try to force your contest date. I don't recommend shooting for symbolic dates unless you've left yourself much more time than you need.

Something really common I've seen with respect to timing is \emph{announcing the contest before the test itself is ready}. It shocks me that this even has to be said. If you aren't ready to run the contest in some form, don't run it to begin with! In fact, I would recommend that you refrain from setting any dates until you've finished compiling the test.

Generally you'll want three to six months to assemble the problems, depending on how much attention to detail you want to put inside your test. It may take longer if you're working out the kinks of your contest format. It took us about eight months to finalize the MAT, though a large reason for that is because we had a pretty far release date and kept revisiting the problems before then.

The format for the MAT took about two weeks to finalize after I came up with it, though we had a more liberal idea for tiebreakers for a while. Because our contest format was so conservative, it was finalized relatively quickly, but I recommend that organizers intending to write contests with more rounds (such as team rounds) give serious thought to time restrictions, format, and the difficulty spread. 

When it comes to logistics, the primary concern is not how long it will take you to get things done. It's getting them done to begin with.

If you plan on making a website, make sure that at least one of your \emph{core members} actually have experience working with a Javascript framework.
If this is not the case, React is by far the most popular framework in the math contest community.
\footnote{The stack I personally use is Node JS + React JS + Next JS + Tailwind CSS, and it's also what we used for MAT. We used Firebase for MAT, but I cannot personally recommend it one way or another because I haven't used it myself.

If you don't plan on making a website, do not plan on getting sponsors. Google Sites and the like are incredibly unprofessional, and I'd imagine it'd tank your chances at most significant sponsorships.

You should bake in at least two months when it comes to sponsor emails; the turnaround can vary drastically, and that's about how long you'll need to finalize your sponsor list.
Every student-run contest so far, including ours, has gotten more sponsors during the advertisement period.
I think this is not ideal for the organizers, though it is workable. It creates a logistics time crunch, which is really no fun.

Empirically, people have given themselves less time than they actually needed for sponsors.
So start about a month earlier than you think you'd need to at the latest.

(Please don't ask me about backend, I don't know backend.)}
Learn it while you still have the time to.

\section{Testsolving and final checks}

You should get a number of people with different skill levels \emph{that have had minimal contact with the test} involved in the test-solving process. Test-solving is a process that means different things to different people, and no two groups use test-solving for the same purposes. Here are some common ones.
\begin{enumerate}
    \item Validity checking -- checking that all of the problems are correctly stated, the answers are what the contest-holders think they are, etc.
    \item Quality checking -- checking that the individual problems are good and the composition is good.
    \item Difficulty checking -- checking that the test is reasonably hard to be interesting to most of the target audience, and more commonly, that it is not too hard for the target audience. This is a lot harder, especially for less established contests, because they aren't exactly sure who their target audience is. 
\end{enumerate}
Whatever the case you should \emph{never use testsolvers as a substitute for test-setters}. If you want help with the test or even individual problems, do not get testsolvers. Testsolvers are supposed to be the final check before you push a contest to production (i.e. print the papers).

\subsection{Common mistakes}
I recommend you read \href{https://www.dennisc.net/resources/careful.pdf}{Careful!} at the very least. If you work through the problems, you'll have a better idea of what I'm trying to communicate.

These mistakes are why you get people from outside to perform thorough \emph{validity checks}. Some of these are so obvious that it almost seems impossible that an entire group of people missed it.

Here are some mistakes that MAC has made. The bolded parts are added fixes. All problems mentioned will be spoiled, so if you want to try them first, do not read the rest of this subsection.

Problems are from the 2020 JMC 10 and the Summer 2021 MAT.

\begin{exam}[JMC 2020/11]
    Jack has eight sticks of different lengths in the set $\{1,2,3,\dots,7,8\}$. How many nonempty subsets of these eight sticks can Jack choose so the range of the lengths is at most $4$ meters?
\end{exam}

There is nothing wrong with the formulation of the problem. But to get our answer, all of us did casework on the range, which we wrongly assumed had to be between $1$ and $4$ (inclusive). It turned out that we forgot about sets with size $1$, which had range $0$. Thus the correct answer was never on the test! Interestingly enough, none of the people who submitted pointed this error out and we only realized a year later.

There's a number of takeaways to be had here, but I think the most important is you shouldn't solve and write up at the same time, particularly for casework problems. Sometimes the elegance or conciseness of a solution can disguise the fact that it doesn't really cover every case properly.

\begin{exam}[MAT 2021/5]
    Let $a, b,$ and $c$ be \emph{distinct} positive integers with $a + b + c = 10.$ Then there exists a quadratic polynomial $p$ satisfying $p(a) = bc,$ $p(b) = ac,$ and $p(c) = ab.$ Find the maximum possible value of $p(10).$
\end{exam}

In short, the problem is solved by doing some algebraic manipulation and ending up with $p(10)=ab+bc+ca$.

There is a procedure that contestants go through every time they solve a ``find maximum'' problem. First, they find the upper bound, and second, they verify it exists and makes sense. They also make sure to watch out for redundant information.

As a problem-writer, even if you are \emph{confident} in your answer, you should still substitute your variables into the equation and make sure the result makes sense. For \emph{integers} $a$, $b$, $c$, $ab+bc+ca$ is maximized when $a=3$, $b=3$, and $c=4$ (or some permutation thereof).

But consider the common mistake made in AIME I 2020/11. If $a=b$, then there is a level of redundancy as the conditions reduce to $p(a)=ac,$ $p(a)=ac,$ and $p(c)=a^2$. Since two points do not uniquely determine a quadratic, $p(10)$ could have been anything. Some contestants did not notice our mistake and put $33$, which they were awarded points for. This is why the ``distinct'' condition was necessary, and the only reason we didn't catch this mistake is because we never checked our construction \emph{from the beginning}.

In both of these problems, our issues were not related to the main idea of the problem. Rather, it was because we did not check details closely enough. This is likely because many of us got conditioned into evaluating the merit of a problem, rather than the contents, when thinking about it as we grow more familiar with our problem base. Outside perspectives are not painted with such bias, which is why it is important to seek and consider them.


\end{document}
